\documentclass{article}

\usepackage[utf8]{inputenc}
\usepackage[english]{babel}
\usepackage{listings}
\usepackage{xcolor}

\usepackage{fancyhdr}
\pagestyle{fancy}
\fancyhf{}
%\renewcommand{\headrulewidth}{0.5pt}
\lhead{Novosibirsk State University}
\rhead{\thepage}
\fancyheadoffset{-0.0\textwidth}


\hoffset=-3.5cm
\textwidth=19cm
\voffset=-2cm
\textheight=24cm
%\marginparwidth=0cm
%\marginparsep=0cm

\title{ACM Team Reference}
\author{Novosibirsk SU}

\lstset{ %
    backgroundcolor=\color{white},   % choose the background color; you must add \usepackage{color} or \usepackage{xcolor}
    basicstyle=\footnotesize,        % the size of the fonts that are used for the code
    breakatwhitespace=false,         % sets if automatic breaks should only happen at whitespace
    breaklines=true,                 % sets automatic line breaking
    captionpos=t,                    % sets the caption-position to bottom
    commentstyle=\color{green},      % comment style
    deletekeywords={...},            % if you want to delete keywords from the given language
    escapeinside={\%*}{*)},          % if you want to add LaTeX within your code
    extendedchars=true,              % lets you use non-ASCII characters; for 8-bits encodings only, does not work with UTF-8
    %  frame=single,                 % adds a frame around the code
    keepspaces=true,                 % keeps spaces in text, useful for keeping indentation of code (possibly needs columns=flexible)
    %keywordstyle=\color{blue},      % keyword style
    language=C++,                    % the language of the code
    morekeywords={*,...},            % if you want to add more keywords to the set
    numbers=right,                   % where to put the line-numbers; possible values are (none, left, right)
    numbersep=-5pt,                  % how far the line-numbers are from the code
    numberstyle=\tiny\color{gray},   % the style that is used for the line-numbers
    rulecolor=\color{black},         % if not set, the frame-color may be changed on line-breaks within not-black text (e.g. comments (green here))
    showspaces=false,                % show spaces everywhere adding particular underscores; it overrides 'showstringspaces'
    showstringspaces=false,          % underline spaces within strings only
    showtabs=false,                  % show tabs within strings adding particular underscores
    stepnumber=1,                    % the step between two line-numbers. If it's 1, each line will be numbered
    stringstyle=\color{mymauve},     % string literal style
    tabsize=4,                       % sets default tabsize to 24spaces
    title=\lstname,                  % show the filename of files included with \lstinputlisting; also try caption instead of title
    basicstyle=\footnotesize\ttfamily,
    keywordstyle=\bfseries\color{green!40!black},
    %commentstyle=\itshape\color{purple!40!black},
    commentstyle=\color{purple!40!black},
    identifierstyle=\color{blue},
    stringstyle=\color{orange}
}

\begin{document}

\lstinputlisting{../template.cpp}
\newpage

\lstinputlisting{../aho_corasik.cpp}
\lstinputlisting{../all.cpp}
\lstinputlisting{../bipartite.cpp}
\lstinputlisting{../cartesian_tree.cpp}
\lstinputlisting{../fft_prime.cpp}
\lstinputlisting{../general_matching.cpp}
\lstinputlisting{../global_min_cut.cpp}
\lstinputlisting{../graph_comp.cpp}
\lstinputlisting{../graph_strong_comp.cpp}
\lstinputlisting{../hungarian.cpp}
\lstinputlisting{../hld.cpp}
\lstinputlisting{../min_cost_max_flow.cpp}
\lstinputlisting{../num_theory.cpp}
\lstinputlisting{../range_tree.cpp}
\lstinputlisting{../s_funcs.cpp}
\lstinputlisting{../suff_automata.cpp}
\lstinputlisting{../suffmass.cpp}
\lstinputlisting{../ukkonen.cpp}
\begin{Large}
	\begin{center}
		A List of Useless Equations
	\end{center}
	
	\begin{minipage}{0.5\textwidth}
		\begin{flushleft}
			\begin{enumerate}
				\item $\sum \limits_{0\leq k \leq n} {n-k \choose k} = Fib_{n+1}$
				\item $k{n \choose k}=n{n-1 \choose k-1}$
				\item ${n \choose k}=\frac{n}{k}{n-1 \choose k-1}$
				\item $\sum \limits_{i=0}^n{n \choose i}=2^n, \sum \limits_{i\geq 0}{n \choose 2i}=\sum \limits_{i\geq 0}{n \choose 2i+1}=2^{n-1}$
				\item $\sum \limits_{i= 0}^k \left( -1 \right) ^i{n \choose i}=\left( -1 \right) ^k{n-1 \choose k}$
				\item $\sum \limits_{i= 0}^k{n+i \choose i}= \sum \limits_{i= 0}^k{n+i \choose n} = {n+k+1 \choose k}$
				\item $1{n \choose 1}+2{n \choose 2}+3{n \choose 3}+…+n{n \choose n}=n2^{n-1}$
				\item $1^2{n \choose 1}+2^2{n \choose 2}+3^2{n \choose 3}+…+n^2{n \choose n}=(n+n^2)2^{n-2}$
				\item $\sum \limits_{k=0}^r{m \choose k}{n \choose r-k}={m+n \choose r}$
				\item $\sum_{i = 1}^n{ia^i} = \frac{a(n a^{n + 1} - (n + 1) a^n + 1)}{(a - 1)^2}$
			\end{enumerate}
		\end{flushleft}
	\end{minipage}
	\begin{minipage}{0.5\textwidth}
		\begin{flushleft}
			\begin{enumerate}
				\setcounter{enumi}{9}
				\item $\ n,r \in N, n > r, \sum \limits_{i=r}^n{i \choose r}={n+1 \choose r+1}$
				\item $\sum \limits_{i=0}^k{k \choose i}^2={2k \choose k}$
				\item $\sum \limits_{k=0}^n{n \choose k}{n \choose n-k}={2n \choose n}$
				\item $\sum \limits_{k=q}^n{n \choose k}{k \choose q}=2^{n-q}{n \choose q}$
				\item $\sum \limits_{i=0}^nk^i{n \choose i}=(k+1)^n$
				\item $\sum \limits_{i=0}^n{2n \choose i}=2^{2n-1}+\frac{1}{2}{2n \choose n}$
				\item $\sum \limits_{i=1}^n{n \choose i}{n-1 \choose i-1}={2n-1 \choose n-1}$
				\item $\sum \limits_{i=0}^n{2n \choose i}^2=\frac{1}{2} \left( {4n \choose 2n}+{2n \choose n}^2 \right)$
				\item $\left(\frac{m}{n}\right) \equiv \prod_{i=0}^{k} \left(\frac{m_{i}}{n_{i}}\right) (mod\; p)$
			\end{enumerate}
		\end{flushleft}
	\end{minipage}
	\end{Large}
\begin{large}
	\begin{enumerate}
		\setcounter{enumi}{16}
		\item $\left( \frac{b-(n-1)(k-1)}{n} \right)$ - number of choices n int ids from [1..k], each have dist at least k
		\item $d(n)=(n-1) \cdot (d(n-1)+d(n-2)),\;\; d(0)=1,d(1)=0$ - num of derangements
		\item $a(n)=a(n-1) + (n-1)a(n-2),\;\;a(0) = a(1) = 1$ - num of involutions
		\item $T(n, k)= n \cdot T(n-1, k) - F(n-1, k) \cdot T(n-k-1, k)$ if $n > k$. Otherwise $T(n,k)=n!$ - num of perm with all cycles len $\leq$ k. $F(n, k) = n \cdot (n - 1) \cdot … \cdot (n - k + 1)$
		\item $S(n,k)=(n-1) \cdot S(n-1,k)+S(n-1,k-1), S(0,0)=1,S(n,0)=S(0,n)=0.$ \par
		 $S(n,k)$ - number of perms len n with k cycles. $\sum_{k=0}^{n}S(n,k) = n!$
		\item $s(n,k)=k \cdot s(n-1,k)+s(n-1,k-1), s(0,0)=1,s(n,0)=s(0,n)=0$ \par
		 $s(n,k)$ - number of unordered splits n-elem set on k nonempty subsets
		\item $\sum\limits_{i=1}^{n}F_i=F_{n+2}-1,\;\; \sum\limits_{i=0}^{n-1}F_{2i+1}=F_{2n},\;\; \sum\limits_{i=1}^{n}F_{2i}=F_{2n+1}-1,\;\;
		\sum\limits_{i=1}^{n}F_{i}^{2}=F_nF_{n+1},\;\;
		gcd(F_m, F_n)=F_{gcd(m, n)}$
	\end{enumerate}
\end{large}

\end{document}
